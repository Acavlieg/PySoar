\chapter{Performance indicators}
A proper understanding of the performance indicators prevents misinterpretation and might explain differences with other software. Therefore this chapter shortly elaborates on the methods of calculation.

\section{Distance covered from task [km]}
When this value is not equal to the task distance an outlanding has occurred.

\section{Start time [hh:mm:ss]}
Start time is given in local time.

\section{Finish time [hh:mm:ss]}
Finish time is in local time. Finish times are not shown for outlandings. For those flights only the finish times of completed legs are shown.

\section{Start height [m]}
The height is taken from the gps fix when the startline is crossed. This altitude can be stored from two sources: barometric altitude, gps altitude. 
\begin{itemize}
\item barometric start height:ICAO ISA above the 1013.25 HPa sea level datum
\item gps start height: above the WGS84 ellipsoid
\end{itemize}
If both are present, the barometric altitude is used.

\section{Average rate of climb [$\frac{m}{s}$]}
The average rate of climb is simply calculated using:
\begin{equation}
R{C_{avg}} = \frac{{\Delta {h_{thermal - loss}} + \Delta {h_{thermal - gain}}}}{{{t_{thermal}}}}
\end{equation}

\section{Average cruise speed [$\frac{km}{h}$]}
The average cruise speed is calculated using:
\begin{equation}
{v_{avg - cruise}} = \frac{{{s_{cruise}}}}{{{t_{cruise}}}}
\end{equation}
This velocity is thus the ground speed.

\section{Average thermal speed [$\frac{km}{h}$]}
The average thermal speed is calculated using:
\begin{equation}
{v_{avg - thermal}} = \frac{{{s_{thermal}}}}{{{t_{thermal}}}}
\end{equation}
This velocity is thus the ground speed.

\section{Average cruise distance [km]}
The average cruise distance is calculated using:
\begin{equation}
{s_{avg - cruise}} = \frac{{{s_{cruise}}}}{{{n_{cruises}}}}
\end{equation}

\section{Average L/D [-]}
The average glide ratio is calculated using:
\begin{equation}
L/{D_{avg}} = \frac{{{s_{cruise}}}}{{\Delta {h_{cruise}}}}
\end{equation}

\section{Excess distance covered [\%]}
The Excess distance covered is calculated as:
\begin{equation}
ED = \frac{{{s_{cruise}} + {s_{thermal{\kern 1pt} drift}}}}{{{s_{task}}}} \cdot 100\% 
\end{equation}
Wherein the thermal drift is the linear distance between the entrance point and the exit point of the thermal.

\section{XC speed [$\frac{km}{h}$]}
The XC speed is calculated using:
\begin{equation}
{v_{xc}} = \frac{{{s_{task}}}}{{{t_{task - flown}}}}
\end{equation}

\section{Percentage turning [\%]}
This parameter is based on thermal time ${{t_{thermal}}}$ and cruise time ${{t_{cruise}}} $, leading to:
\begin{equation}
{T_{thermal}} = \frac{{{t_{thermal}}}}{{{t_{thermal}} + {t_{cruise}}}} \cdot 100\% 
\end{equation}

\section{Height loss during circling [\%]}
When in thermal mode, two consecutive gps fixed can have either a positive or a negative altitude difference. Positive altitude gains are stored in $\Delta {h_{thermal-gain}}$ while negative altitude gains are stored in ${\Delta {h_{thermal-loss}}}$. The height loss during circling is calculated as:
\begin{equation}
{H_{loss - thermal}} = \frac{{\Delta {h_{thermal - loss}}}}{{\left| {\Delta {h_{thermal - loss}}} \right| + \Delta {h_{thermal - gain}}}} \cdot 100\% 
\end{equation}
This parameter can be interpreted as a measure for how well the thermal has been centered.